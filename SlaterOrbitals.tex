\documentclass[%
aip,%
jcp,%
amsmath,amssymb,%
%preprint,%
reprint,%
]{revtex4-1}

\usepackage[utf8]{inputenc}

\newcommand*{\radint}{\int_0^{\infty}}
\newcommand*{\Ncal}{\mathcal{N}}
\newcommand*{\spheric}{(r,\theta,\varphi)}			% (r, theta, phi)
\newcommand*{\angulaire}{(\theta,\varphi)}			% (theta, phi)


\begin{document}

\title{Slater atomic orbitals and slater's rule}

\author{Germain Salvato Vallverdu}
\email{germain.vallverdu@univ-pau.fr}
\affiliation{Université de Pau et des Pays de l'Adour, IPREM - ECP, CNRS UMR 5254}

\begin{abstract}
A short review on Slater type orbitals and the Slater's rule.
\end{abstract}

\maketitle

\section{Orbitales de Slater}

\subsection{Expression générale :}
%
\begin{equation*}
    \Psi_{n,\ell,m_\ell}\spheric = \Ncal r^{n-1} \exp\left( -\frac{Z^* r}{n\,a_o} \right)
    Y_\ell^{m_\ell}\angulaire
\end{equation*}
%
Densité de probabilité de présence radiale :
%
\begin{equation*}
    \mathcal{D}_{n,\ell}(r) = r^2 |\Psi_{n,l}(r)|^2 = \Ncal^2 r^{2n} 
        \exp\left( -\frac{2Z^*r}{n\,a_o} \right)
\end{equation*}
%
Calcul du coefficient de normalisation :
%
\begin{align*}
    \radint \mathcal{D}_{n,\ell}(r) dr  & = 1 \\
    & = \radint \Ncal^2 r^{2n} \exp\left( -\frac{2Z^*r}{n\,a_o} \right) \\
    & = \Ncal^2 \radint r^{2n} \exp\left( -\frac{2Z^*r}{n\,a_o} \right) \\
    & = \Ncal^2 \, (2n)! \, \left(\frac{na_o}{2Z^*}\right)^{2n+1}
\end{align*}
%
d'où
%
\begin{equation*}
    \Ncal =\frac{1}{\sqrt{(2n)!}} \left( \frac{2Z^*}{na_o} \right)^{n+1/2} 
\end{equation*}

Expression analytiques des fonctions radiales des premières valeurs de n :
%
\begin{align*}
    R_{1,\ell}(r) & =  2\left( \frac{Z^*}{a_o} \right)^{3/2} 
        \exp\left( -\frac{Z^* r}{a_o} \right) \\
    R_{2,\ell}(r) & =  \frac{1}{\sqrt{24}} \left( \frac{Z^*}{a_o} \right)^{5/2} r 
        \exp\left( -\frac{Z^* r}{2a_o} \right) \\
    R_{3,\ell}(r) & =  \frac{1}{12\sqrt{5}} \left( \frac{2Z^*}{3a_o} \right)^{7/2} r^2
        \exp\left( -\frac{Z^* r}{3a_o} \right) \\
\end{align*}

\subsection{Rayon orbitalaire}

On définit le rayon orbitaire comme la valeur pour laquelle la densité de probabilté de
présence radiale, $\mathcal{D}_{n,\ell}(r)$, est maximale. 

Pour trouver les maxima de $\mathcal{D}_{n,\ell}(r)$ on calcule la dérivée première :

\begin{align*}
    \frac{d\mathcal{D}_{n,\ell}(r)}{dr} & = \frac{d}{dr} \Ncal^2 r^{2n} 
        \exp\left( -\frac{2Z^*r}{n\,a_o} \right) \\
    & = \Ncal^2 \left[ 2n r^{2n-1} - \frac{2Z^*}{na_o}r^{2n} \right] 
    \exp \left( -\frac{2Z^*r}{na_o} \right) \\
    & = 2\Ncal^2 r^{2n-1}\left[ n - \frac{Z^*r}{na_o} \right] 
    \exp \left( -\frac{2Z^*r}{na_o} \right)
\end{align*}

La dérivée est nulle pour $r_1$=0 et pour une valeur $r_2$ qui annule la partie entre
crochets.
%
\begin{equation*}
    r_2 = \frac{n^2}{Z^*} a_o
\end{equation*}
%
Pour des valeurs de $r\in$, le signe de la dérivée est donnée par la partie entre
crochets. La dérivée est positive pour des valeurs de $r$ inférieure à $r_2$ et négative
pour des valeurs de $r$ supérieure à $r_2$. On a donc le tableau de variation suivant :

\renewcommand{\arraystretch}{1.5}
\begin{tabular}{c|lcccl}
    $r$ & 0 & & $\frac{n^2}{Z^*} a_o$ & & $+\infty$ \\
    \hline
    $d\mathcal{D}_{nl}/dr$ & & + & 0 & - & \\
    \hline
    $\mathcal{D}_{nl}$ &
\end{tabular}

\section{Intégrales}

\begin{equation*}
    \int x\,e^{ax} dx = \left( \frac{x}{a} - \frac{1}{a^2} \right) e^{ax}
\end{equation*}

\begin{equation*}
    \int x^2\,e^{ax} dx = \left( \frac{x^2}{a} - \frac{2x}{a^2} + \frac{2}{a^3} \right) e^{ax}
\end{equation*}

\begin{equation*}
    \int x^3\,e^{ax} dx = \left(\frac{x^3}{a} - \frac{3x^2}{a^2} + \frac{6x}{a^3} -
    \frac{6}{a^4} \right) e^{ax}
\end{equation*}

\begin{equation*}
    \int x^4\,e^{ax} dx = \left( \frac{x^4}{a} - \frac{4x^3}{a^2} + \frac{12x^2}{a^3} 
    - \frac{24x}{a^4} + \frac{24}{a^5} \right) e^{ax}
\end{equation*}

\begin{equation*}
    \int x^n \, e^{ax} dx = \frac{1}{a} x^n \, e^{ax} - 
        \frac{n}{a} \int x^{n-1} e^{ax} dx
\end{equation*}

\begin{equation*}
    \int_0^{\infty} x^n \, e^{-ax} dx = \frac{n!}{a^{n+1}}
\end{equation*}


\end{document}
